\section{Introduction}
\label{sec_introduction}
%\section{Introduction}
% u can reorganized some content herer
%
%The following provides a short summary of related information in these papers.
%There are two different analysis of non-orthogonal multiple access (NOMA) and successive interference cancellation (SIC). One analyzes SIC from original Shannon capacity and anther start from modulation scheme and packet error rate (PER). The analysis starting from Shannon capacity is mainly proposed from DOCOMO Japan~\cite{cite_docomo1, cite_docomo2, cite_docomo3}. Another one from the aspect of modulation and PER is proposed from Bell Lab~\cite{cite_bell1,cite_bell2}.
%

Briefly, by the previous effort in literature survey, we summarize related work
as follows.
In~\cite{cite_docomo1}, the authors compare the performance between orthogonal 
frequency-division multiple access (OFDMA) and NOMA. They in particular
investigate successive interference cancellation (SIC) and state that
SIC receivers should follow the rule that the optimal order for decoding
is in the order of the increasing $\frac{|h|^2}{N}$. An NOMA/Multiple-Input
Multiple-Output (MIMO) scheme is proposed to achieve further capacity gain.
In~\cite{cite_docomo2}, the authors investigate the enhancement of the
cell-edge user throughput by using SIC in  the cellular downlink. They
propose an optimization method that can balance the throughput of cell-edge
users and interior users. The work in~\cite{cite_docomo3} is the extension
of~\cite{cite_docomo2}, where the authors investigate the use of FFR and
weighted PF-based multi-user scheduling with SIC in the cellular downlink.
They show that NOMA can help achieve user fairness.
%
%The detailed description of related work from DOCOMO Japan will be given in Section~\ref{sec_phy} and Section~\ref{sec_mac}.
%

Related work in~\cite{cite_bell1} not only considers the ideal channel
capacity but also analyzes SIC performance with respect to modulation schemes
and packet error rate (PER). A resource allocation algorithm is proposed to
leverage the spatial gain and the simulation results show that
proposed algorithm using NOMA with SIC achieves more than 100 percent throughput
gain comparing OFDMA. In~\cite{cite_bell2}, the authors investigate SIC and
proposed an iterative SIC receiver architecture with combined pilot/date based
channel estimation for efficient decoding of NOMA signals. The simulation
results show iterative SIC receivers reach low PER with weak SNR under a
reasonable complexity.

%the authors analyze the iterative SIC receiver structure and present simulation results of their receiver structure with respect to PER.
%The simulation results are also given in the aspect of PER. %The detailed description is also in the following paragraphs.

%In order to compare two different analysis, the following paragraphs are divided into two sections for different topics (PHY layer and MAC layer). The PHY layer in Section~\ref{sec_phy} shows the NOMA/SIC from two different aspects: starting from Shannon capacity and starting from modulation/PER. The MAC layer in Section~\ref{sec_mac} discusses resource allocation and scheduling problems. We make a conclusion in Section~\ref{sec_conclusion} and describe our future direction in Section~\ref{sec_futureWork}. 